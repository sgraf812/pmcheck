
%%%%%%%%% HERE THE MACROS START
% Highlight changes and keyword undone
% \newcommand{\highlight}[1]{\colorbox{lightgray}{\ensuremath{#1}}}

\newcommand{\highlight}[1]{\setlength{\fboxsep}{2pt}\colorbox{lightgray}{\ensuremath{#1}}} % less height

\newcommand{\UNDONE}{\begin{color}{red}{\bf UNDONE }\end{color}}
\newcommand{\DONE}{\begin{color}{green}{\bf DONE}\end{color}}
\newcommand{\NOTE}[1]{\bf #1}
\newcommand{\TODO}[1]{{\bf{\begin{color}{red}{TODO: }\end{color} #1}}}

% Useful macros that are often needed
%\newcommand\WILD{\mbox{@_@}}
\newcommand\EMPTY{\varnothing}
\newcommand\PS{\overline{p}}
\newcommand\QS{\overline{q}}
\newcommand\VS{\overline{v}}

\newcommand{\XS}[1]{\vec{x}_{#1}}
\newcommand{\YS}[1]{\vec{y}_{#1}}
\newcommand{\ubot}[1]{#1_{\bot}}    % x_bot

\newcommand{\typesig}{\colon}          % type equality
\newcommand{\typeeq}{\sim}          % type equality
\newcommand{\termeq}{\approx}       % positive term equality
\newcommand{\ntermeq}{\not\approx}  % negative term equality


% All functions in the document
\newcommand\MISSING{\mathit{missing}}
\newcommand\USEFUL{\mathit{useful}}
\newcommand\USED{\mathit{used}}
\newcommand\UNUSED{\mathit{unused}}
\newcommand\REDUNDANT{\mathit{redundant}}
\newcommand\MISSINGROW{\mathit{missing_0}} %_1} %\_row}}
\newcommand\EXTENDS{\mathit{extends}}
\newcommand\COVERS{\mathit{covers}}
\newcommand\EXTENDSSET{\mathit{extends\_set}}
\newcommand\COVERSSET{\mathit{covers\_set}}

% Matching predicate
\newcommand{\MATCH}{\mathit{match}}
\newcommand{\TRUE}{\mathcal{T}}
\newcommand{\FALSE}{\mathcal{F}}

% Actual function versions (with parameters)
\newcommand{\MISSINGROWF}[1]{\MISSING\: #1}
\newcommand{\EXTENDSF}[2]{\EXTENDS \: #1\: #2}
\newcommand{\COVERSF} [2]{\COVERS  \: #1\: #2}
\newcommand{\COVERSP} [2]{\COVERS' \: #1\: #2} % covers prime
\newcommand{\MATCHF}[2]{\MATCH \: (#1, #2)}

% s_u, s_c, s_bot
\newcommand{\setcovered}[1]{C_{#1}}      % Set of value abstractions covered by a pattern
\newcommand{\setuncovered}[1]{U_{#1}}    % Set of value abstractions un-covered by a pattern
\newcommand{\setdiverges}[1]{D_{#1}}     % Set of value abstractions that diverge when matched

% Not sure that is useful yet
\newcommand\FORCES{\mathit{forces}}
\newcommand{\FORCESF}[2]{\FORCES\: #1\: #2}

\newcommand{\litc}[1]{\ensuremath{\{#1\}}}

% To represent predicates
\newcommand\PP{\mathcal{P}}
\newcommand\QQ{\mathcal{Q}}
\newcommand\RR{\mathcal{R}}

% To highlight them
\newcommand\HPP{\highlight{\PP}}
\newcommand\HQQ{\highlight{\QQ}}
\newcommand\HRR{\highlight{\RR}}

% Typechecking
\newcommand{\CONSISTENT}{\mathit{consistent}}

\newcommand{\OutsideIn}[1]{{\rm {\sc OutsideIn}(#1)}\xspace}

% extends', covers' and forces' names
\newcommand\EName[1]{\textsc{\small{(E$\_$#1)}}}
\newcommand\CName[1]{\textsc{\small{(C$\_$#1)}}}
\newcommand\FName[1]{\textsc{\small{(F$\_$#1)}}}

\newcommand\False{\mathit{False}}
\newcommand\True{\mathit{True}}


\newcommand{\emptyvec}{\epsilon}   % Empty list vector for the vec notation
\newcommand{\gtup}[2]{#2 \vdash #1}


% Interpretation
\newcommand{\tinterp}[1]{\mathcal{T} \llbracket #1 \rrbracket}
\newcommand{\interp}[1]{\llbracket #1 \rrbracket}
\newcommand{\ginterp}[2]{{\llbracket #1 \rrbracket}_{#2}}
\newcommand{\minterp}[1]{\ginterp{#1}{M}}
\newcommand{\pinterpm}[3]{\interp{#1}^{#2}(#3)}
\newcommand{\qinterp}[1]{\interp{#1}}
\newcommand{\tauinterp}[1]{\ginterp{#1}{\overline{\tau}}}
\newcommand{\einterp}[1]{E\interp{#1}}

\newcommand{\ftv}[1]{\mathit{ftv}(#1)}

\newcommand{\ruleform}[1]{\fbox{$#1$}}
\newcommand{\rulename}[1]{\textsc{[#1]}}

\newcommand{\alg}[2]{\mathit{alg}(#1, #2)}
\newcommand{\algorithm}[2]{\mathit{alg}(#1, #2)} % replace appearances of this with \alg{}{}
\newcommand{\triple}[3]{\langle #1, #2, #3 \rangle}

\newcommand{\SC}{\setcovered{}}
\newcommand{\SB}{\setdiverges{}}
\newcommand{\SU}{\setuncovered{}}

\newcommand{\tricup}{\cup^\lrangle}
\newcommand{\bigtricup}{\bigcup^\lrangle}

\newcommand{\lrangle}{{\langle\rangle}}
\newcommand{\tmap}{\mathit{map}^\lrangle}

\newcommand{\ticket}[1]{\href{https://ghc.haskell.org/trac/ghc/ticket/#1}{\##1}}



\newcommand{\gk}[1]{\begin{color}{red}{\bf GK:} #1\end{color}}
\newcommand{\tom}[1]{\begin{color}{blue}{\bf TOM:} #1\end{color}}
\newcommand{\simon}[1]{\begin{color}{purple}{\bf SLPJ:} #1\end{color}}
\newcommand{\dv}[1]{\begin{color}{green}{\bf DV:} #1\end{color}}

% \newcommand{\gk}[1]{}
% \newcommand{\tom}[1]{}
% \newcommand{\simon}[1]{}
% \newcommand{\dv}[1]{}


\newcommand{\TRANSLATE}{\mathit{desugar}}
\newcommand{\TRANSLATEF}[1]{\TRANSLATE\: #1}
\newcommand{\TRANSLATEP}[1]{\mathit{dsP}\: #1}
\newcommand{\TRANSLATEG}[1]{\mathit{dsG}\: #1}

\newcommand{\TRANSLATEV}{\mathit{view}}
\newcommand{\TRANSLATEVF}[2]{\mathit{\TRANSLATEV\:{#1}\:{#2}}}

\newcommand{\TRANSLATES}{\mathit{dsPs}}

\newcommand{\HsPat}{\mathbb{P}}
\newcommand{\HsG}{\mathbb{G}}
%%%%%%%%%%%%%%%%%%%%%%%%%%%%%%%%%%%%%%%%%%%%%%%%%%%%%%%%%%%%%%%%%
% FROM APRIL

% THERE IS A PROBLEM

% This is the stuff from 2019
\newcommand{\ty}[1]{\textsf{#1}\xspace}
\newcommand{\Pat}{\ty{Pat}}
\newcommand{\Grd}{\ty{Grd}}
\newcommand{\Con}{\ty{Con}}
\newcommand{\Var}{\ty{Var}}
\newcommand{\Expr}{\ty{Expr}}
\newcommand{\Type}{\ty{Type}}
\newcommand{\Kind}{\ty{Kind}}
\newcommand{\TyCt}{\ty{TyCt}}
\newcommand{\grdlet}[2]{\textsf{let}\;#1\,\textsf{=}\,#2}
\newcommand{\grdbang}[1]{\textsf{!}#1}
\newcommand{\grdcon}[2]{#1 \leftarrow #2}
\newcommand{\ctlet}[2]{#1\termeq#2}
\newcommand{\ctcon}[2]{#1 \leftarrow #2}
\newcommand{\genconapp}[4]{#1\;\overline{#2}\;\overline{#3}\;\overline{#4}}
% GrdTree Gdt
\newcommand{\Gdt}{\ty{Gdt}}
\newcommand{\gdtrhs}[1]{\texttt{Rhs} \; #1}
\newcommand{\gdtseq}[2]{#1; #2}
\newcommand{\gdtguard}[2]{\texttt{Guard} \; #1 \; #2}
% AnnTree Ant
\newcommand{\Ant}{\ty{Ant}}
\newcommand{\antred}[1]{\texttt{InaccessibleRhs} \; #1}
\newcommand{\antrhs}[1]{\texttt{AccessibleRhs} \; #1}
\newcommand{\antseq}[2]{#1; #2}
\newcommand{\antdiv}[1]{\texttt{MayDiverge} \; #1}
% Checking functions
\newcommand{\unc}[2]{\mathcal{U}(#1,#2)}
\newcommand{\ann}[2]{\mathcal{A}_\Gamma(#1,#2)}
\newcommand{\values}[2]{\mathcal{M}(#1,#2)}
\newcommand{\blah}[2]{\mathcal{I}(#1,#2)}

\newcommand{\ctxt}[2]{#1 \triangleright #2}
\newcommand{\addinert}[2]{#1 \oplus #2}
\newcommand{\inh}[2]{#1 \vdash #2}

\newcommand{\evdnce}[2]{#1 \triangleright #2}
\newcommand{\vtup}[3]{#1 \vdash #2 \deltasep #3}
\newcommand{\vtupnew}[3]{#1 \vdash #2 \deltasep #3}
\newcommand{\sat}[1]{\vdash_\textsc{Sat} #1}
\newcommand{\unsat}[1]{\nvdash_\textsc{Sat} #1}
% Different views on "Delta":
% - The set of values it approximates. Ties to abstract interpretation
% - A boolean formula, about which you can ask if it's satisfiable (= non-empty)
% - Refinement type of the value vector/free variables of the pattern 
% - Syntactically: A disjunction of (possibly empty) constraint lists
% The latter is what we use for syntax, the interpretation is probably in terms of refinement types or so...
\newcommand{\false}{\times}
\newcommand{\true}{\checked}

%% \newcommand{\deltasep}{\:[\hspace{-1pt}]\:}
\newcommand{\deltasep}{\:\triangleright\:}

\newcommand{\MAP}{\mathit{map}}
\newcommand{\TAIL}{\mathit{tail}}
\newcommand{\CONS}{\mathit{ucon}}
\newcommand{\ZIPCON}{\mathit{kcon}}

\newcommand{\CHECK}{{\mathcal{P}}}
\newcommand{\COVERED}{{\mathcal\setcovered{}}}     %% \mathit{covered}}
\newcommand{\UNCOVERED}{{\mathcal\setuncovered{}}} %% \mathit{uncovered}}
\newcommand{\ELIMINATED}{{\mathcal\setdiverges{}}} %% \mathit{eliminated}}

%% Parenthesized versions
\newcommand{\covered}[2]{\COVERED\:(#1)\;(#2)}
\newcommand{\uncovered}[2]{\UNCOVERED\:(#1)\;(#2)}
\newcommand{\eliminated}[2]{\ELIMINATED\:(#1)\;(#2)}

%% Unparenthesized versions
\newcommand{\ucovered}[2]{\COVERED\:#1\:#2}
\newcommand{\uuncovered}[2]{\UNCOVERED\:#1\:#2}
\newcommand{\ueliminated}[2]{\ELIMINATED\:#1\:#2}


\newcommand{\cons}[2]{#1\:#2}

\newcommand{\ps}{\vec{p}}
\newcommand{\as}{\vec{a}}

\newcommand{\diverges}[1]{#1 = \bot}
